\documentclass{article}
\usepackage{titlepage}
\usepackage[francais]{babel}
\usepackage[T1]{fontenc}
\usepackage[utf8]{inputenc}
\usepackage{graphicx}
\graphicspath{{image/}}


\title{Tests unitaires}
\subtitle{TP1}
\dateremise{le 2 Octobre 2018}
\author{Billy Bouchard}{Jacob Dorais}{}
\prof{Hiba Bagane}


\begin{document}
\maketitle
\section*{Question}
\subsection*{Quelle méthode avez vous choisi pour empêcher la classe JsonClient d’exécuter
  les vrais appels API durant vos tests unitaires?}
Nous avons choisis d'utiliser un stub de la method fetch de l'objet nodeFetch.
Il s'agit d'une modification de la méthode qui remplace la définition utilisée dans le code de base.
Un stub permet donc de remplacer complètement l'implementation de la méthode fetch de l'objet nodeFetch.
On peux alors choisir de renvoyer les réponse que l'on veux

\end{document}
