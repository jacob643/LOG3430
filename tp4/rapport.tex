\documentclass[11pt]{article}
\usepackage{forest}
\usepackage{graphicx}
\graphicspath{{image/}}
\usepackage[utf8]{inputenc}
\usepackage{titlepage}
\title{TP4}
\subtitle{Test Basés sur les états}
\dateremise{le lundi 12 novembre}
\author{Jacob Dorais}{Billy Bouchard}{Gr 02}
\prof{Hiba Bagane}
\forestset{qtree/.style={for tree={parent anchor=south, child anchor=north,align=center,inner sep=0pt}}}
\usepackage{geometry}
\geometry{
	right=20mm,
	left=20mm
}
\renewcommand{\baselinestretch}{1.6}
\usepackage{tikz}
\usepackage[francais]{babel}
\usepackage[T1]{fontenc}
\usepackage{float}
\begin{document}
\maketitle
\section{Choix des États}
Nous avons décider de faire un un total de 3 états soit : un état vide, un état avec un seul élement,un element lorsque la file contient plus que 3 éléments.
Le choix de ces états ces surtout basée sur les propriété de first et last de la file.
En effet, dans le premier état, first et last ne sont pas défini et accédé a leur valeur retourne une erreur.
Dans le second état, ils sont défini, mais on la même valeur.
Dans le dernier état, ils sont différents(pointe vers 2 élément différents).
Dans le diagramme des états ci-dessous, on peut voir toutes les transitions entre les états.
Il est important de noter que seulement 2 méthodes peuvent modifier les états de la file soit enqueue et dequeue
\subsection{Diagramme des États}
\begin{figure}[h]
  \begin{center}
    \begin{tikzpicture}[r/.style={rectangle,draw=black, minimum size=5mm},
        c/.style={circle,draw=black,fill=black, minimum size=5mm}]

      \node[c] (c) at (-9, -3) {};

      \node[r] (e) at (-8, 0) {
        \begin{tabular}{l}
          Empty        \\ \hline
          first = null \\
          last = null  \\
          n = 0
        \end{tabular}
      };
      \node[r] (p) at (-4, 0) {
        \begin{tabular}{l}
          One Node     \\ \hline
          first = last \\
          n = 1
        \end{tabular}
      };
      \node[r] (f) at (0,0) {
        \begin{tabular}{l}
          Multi Node        \\ \hline
          first $\neq$ last \\
          n $>$ 1
        \end{tabular}
      };

      \draw[->] (c.north) -- (e.south) node [midway, left] (text) {create};
      \draw[->] (e.east) -- (p.west) node [midway, above, sloped] (text) {enqueue()};
      \draw[->] (e) to [out=200, in=160, looseness=4] node [midway, left] (text) {dequeue()} (e);

      \draw[->] (p.south) to [out=270,in=270] node [midway, below] (text) {dequeue()} (e.south) ;
      \draw[->] (p.east) -- (f.west) node [midway, above, sloped] (text) {enqueue()};
      \draw[->] (f.north) to [out=90,in=90] node [midway, above] (text) {dequeue() $[n=2]$ } (p.north);
      \draw[->] (f) to [out=20, in=-20, looseness=4] node [midway, right] (text) {
        \begin{tabular}{l}
          enqueue(), \\
          dequeue() $[ n > 2 ]$
        \end{tabular}
      } (f);
    \end{tikzpicture}
  \end{center}
  \caption{Diagramme des états}
\end{figure}

\newpage
\subsection{Diagramme des passage d'états}
dans le schémas à la figure 2, on peut observer que l'arbre contient six branches finales en partant de son initialisation.
\begin{figure}[!h]
  \begin{center}
    \begin{forest}
      [create
          [Empty
              [One Node, edge label={node[midway,left,font=\scriptsize]{enqueue()}}
                  [Multi Node,edge label={node[midway,left,font=\scriptsize]{enqueue()}}
                      [Multi Node,edge label={node[midway,left,font=\scriptsize]{\begin{tabular}{l}
                                  enqueue() \\
                                  dequeue() $[n>2]$
                                \end{tabular}}} ]
                      [One Node,edge label={node[midway,right,font=\scriptsize]{dequeue() $[n=2]$}} ]
                  ]
                  [Empty, edge label={node[midway,right,font=\scriptsize]{dequeue()}}]
              ]
              [Empty, edge label={node[midway,right,font=\scriptsize]{dequeue()}}]
          ]
      ]
    \end{forest}
  \end{center}
  \caption{Passage des états}
\end{figure}
on obtient alors les 5 tests suivant :
\begin{enumerate}
  \item create$\rightarrow$[Empty].dequeue$\rightarrow$[Empty]
  \item create$\rightarrow$[Empty].enqueue$\rightarrow$[One Node].dequeue$\rightarrow$[Empty]
  \item create$\rightarrow$[Empty].enqueue$\rightarrow$[One Node].enqueue$\rightarrow$[Multi node].dequeue$\rightarrow$[One Node]
  \item create$\rightarrow$[Empty].enqueue$\rightarrow$[One Node].enqueue$\rightarrow$[Multi node].enqueue$\rightarrow$[Multi Node]
  \item create$\rightarrow$[Empty].enqueue$\rightarrow$[One Node].enqueue$\rightarrow$[Multi node].enqueue$\rightarrow$[Multi Node]$\rightarrow$[Multi Node].dequeue$\rightarrow$[Multi Node]
\end{enumerate}

\end{document}
